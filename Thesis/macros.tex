%\usepackage[firstpage]{draftwatermark} %voor het TU Delft watermerk
\usepackage[latin1]{inputenc} 	
\usepackage[english]{babel} % nederlandse teksten
\usepackage{amsmath} %AMS packages voor betere formules
\usepackage{amsfonts}
\usepackage{amssymb}
\usepackage{graphicx} %voor plaatjes: \includegraphics{}
\usepackage{wrapfig} %voor plaatjes naast tekst
\usepackage[headings,cm]{fullpage} %grotere paginasize
\usepackage[margin=1in]{geometry} %custom geometry
\usepackage{float}		%%%%% voor [H]
%\usepackage{url} 	%voor \url{} omgeving, niet gebruikt
\usepackage{multicol} %multicolom dingen
\usepackage{hyperref} 	% voor links in PDF's en meer 

\usepackage{pdfpages}

%\usepackage{titlepic} 	% plaatje op eerste pagina
\usepackage{eso-pic}%Voor TU-bies en blokken
\usepackage{nextpage}			% Advanced nextpage commands

\usepackage{standalone} %voor gefixte \input 
\usepackage{booktabs} %voor mooie tabellen met \toprule \midrule en \bottomrule
\usepackage{verbatim} %voor input text
\usepackage{enumitem} 	% voor mooie enumerate 
%\usepackage[authoryear]{natbib}
%\setdescription{leftmargin=\parindent,labelindent=\parindent}

\usepackage{xcolor}
\colorlet{keyword}{blue!100!black!80}
\colorlet{comment}{green!90!black!90}
%\definecolor{mygreen}{rgb}{0,0.6,0}

\usepackage{listings} %gebruikt voor mooie code-invoer met highlighting
\lstset{	numbers		= left,			%
		numberstyle	= \tiny,		%
		numbersep	= 5pt,			%
		language	= VHDL,			%
		breaklines	= true,			%
		showspaces	= false,		%
		showstringspaces= false}

\lstdefinestyle{vhdl}{
  language     = VHDL,
  basicstyle   = \ttfamily,
  keywordstyle = \color{keyword}\bfseries,
  commentstyle = \color{comment}
}




\usepackage{siunitx}
\sisetup{load-configurations = abbreviations}




\usepackage{multirow} %voor meerdere rijen in tabellen
\usepackage{fix-cm} %%%%%
\newcommand{\matlab}{{\textsc{matlab }}} %MATLAB moet geschreven worden in smallcaps. Dit defineerd het commando \matlab dat dit doet

%\autoref{} dingen
\renewcommand{\tableautorefname}{tabel}
\renewcommand{\figureautorefname}{figuur}
\renewcommand{\chapterautorefname}{hoofdstuk}
\renewcommand{\sectionautorefname}{sectie}
\renewcommand{\subsectionautorefname}{sectie}
\newcommand{\lstnumberautorefname}{Code}
\renewcommand{\subsubsectionautorefname}{sectie}


%big reference:
\newcommand{\maxwell}[1]
{\autoref{#1}: `\nameref{#1}'}



\usepackage{calc}
% een omgeving genaamt DESCRIPTION die werk als een enumerate, maar nu blijft de text op dezelfde indent
\makeatletter
\newcommand{\DESCRIPTION@original@item}{}
\let\DESCRIPTION@original@item\item
\newcommand*{\DESCRIPTION@envir}{DESCRIPTION}
\newlength{\DESCRIPTION@totalleftmargin}
\newlength{\DESCRIPTION@linewidth}
\newcommand{\DESCRIPTION@makelabel}[1]{\llap{#1}}%
\newcommand{\DESCRIPTION@item}[1][]{%
  \setlength{\@totalleftmargin}%
       {\DESCRIPTION@totalleftmargin+\widthof{\textbf{#1 }}-\leftmargin}%
  \setlength{\linewidth}
       {\DESCRIPTION@linewidth-\widthof{\textbf{#1 }}+\leftmargin}%
  \par\parshape \@ne \@totalleftmargin \linewidth
  \DESCRIPTION@original@item[\textbf{#1}]%
}
\newenvironment{DESCRIPTION}
  {\list{}{\setlength{\labelwidth}{0cm}%
           \let\makelabel\DESCRIPTION@makelabel}%
   \setlength{\DESCRIPTION@totalleftmargin}{\@totalleftmargin}%
   \setlength{\DESCRIPTION@linewidth}{\linewidth}%
   \renewcommand{\item}{\ifx\@currenvir\DESCRIPTION@envir
                           \expandafter\DESCRIPTION@item
                        \else
                           \expandafter\DESCRIPTION@original@item
                        \fi}}
  {\endlist}
\makeatother



%Dit is voor een newline na een paragraph. Vind ik mooier. als je dat niet vind moet je het verwijderen.
\makeatletter
\renewcommand\paragraph{\@startsection{paragraph}{4}{\z@}%
  {-3.25ex\@plus -1ex \@minus -.2ex}%
  {1.5ex \@plus .2ex}%
  {\normalfont\normalsize\bfseries}}
\makeatother
%end

%dit is zodat hyperlinks niet blauw zijn.
\hypersetup{
    colorlinks,
    citecolor=black,
    filecolor=black,
    linkcolor=black,
    urlcolor=black
}

\setlength{\parindent}{0cm}

\let\endtitlepage\relax


